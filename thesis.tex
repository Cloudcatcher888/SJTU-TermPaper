%# -*- coding: utf-8-unix -*-
%%==================================================
%% thesis.tex
%%==================================================

% 双面打印
\documentclass[bachelor, fontset=fandol, openany, oneside, submit]{sjtuthesis}
% \documentclass[master, fontset=fandol, review]{sjtuthesis}
% \documentclass[doctor, fontset=fandol, openright, twoside]{sjtuthesis}
% \documentclass[%
%   bachelor|master|doctor,	% 必选项
%   fontset=adobe|windows|fandol|source, % 字体选项,adobe 需要安装 adobe 对应字体才可使用,因此建议使用 source 和 fandol 开源选项
%   oneside|twoside,		% 单面打印,双面打印(奇偶页交换页边距,默认)
%   openany|openright, 		% 可以在奇数或者偶数页开新章|只在奇数页开新章(默认)
%   zihao=-4|5,, 		% 正文字号:小四、五号(默认)
%   review,	 		% 盲审论文,隐去作者姓名、学号、导师姓名、致谢、发表论文和参与的项目
%   submit			% 定稿提交的论文,插入签名扫描版的原创性声明、授权声明 
% ]

% 逐个导入参考文献数据库
\addbibresource{bib/thesis.bib}
% \addbibresource{bib/chap2.bib}

\begin{document}

%% 无编号内容:中英文论文封面、授权页
\include{tex/id}
\maketitle

\makeatletter
\ifsjtu@submit\relax
	\includepdf{pdf/original.pdf}
	\cleardoublepage
	\includepdf{pdf/authorization.pdf}
	\cleardoublepage
\else
\ifsjtu@review\relax
% exclude the original claim and authorization
\else
	\makeDeclareOriginal
	\makeDeclareAuthorization
\fi
\fi
\makeatother


\frontmatter 	% 使用罗马数字对前言编号

%% 摘要
\pagestyle{main}
%# -*- coding: utf-8-unix -*-
%%==================================================
%% abstract.tex for SJTU Master Thesis
%%==================================================

\begin{abstract}

上海交通大学是我国历史最悠久的高等学府之一,是教育部直属、教育部与上海市共建的全国重点大学,是国家 “七五”、“八五”重点建设和“211工程”、“985工程”的首批建设高校。经过115年的不懈努力,上海交通大学已经成为一所“综合性、研究型、国际化”的国内一流、国际知名大学,并正在向世界一流大学稳步迈进。 

十九世纪末,甲午战败,民族危难。中国近代著名实业家、教育家盛宣怀和一批有识之士秉持“自强首在储才,储才必先兴学”的信念,于1896年在上海创办了交通大学的前身——南洋公学。建校伊始,学校即坚持“求实学,务实业”的宗旨,以培养“第一等人才”为教育目标,精勤进取,笃行不倦,在二十世纪二三十年代已成为国内著名的高等学府,被誉为“东方MIT”。抗战时期,广大师生历尽艰难,移转租界,内迁重庆,坚持办学,不少学生投笔从戎,浴血沙场。解放前夕,广大师生积极投身民主革命,学校被誉为“民主堡垒”。

新中国成立初期,为配合国家经济建设的需要,学校调整出相当一部分优势专业、师资设备,支持国内兄弟院校的发展。五十年代中期,学校又响应国家建设大西北的号召,根据国务院决定,部分迁往西安,分为交通大学上海部分和西安部分。1959年3月两部分同时被列为全国重点大学,7月经国务院批准分别独立建制,交通大学上海部分启用“上海交通大学”校名。历经西迁、两地办学、独立办学等变迁,为构建新中国的高等教育体系,促进社会主义建设做出了重要贡献。六七十年代,学校先后归属国防科工委和六机部领导,积极投身国防人才培养和国防科研,为“两弹一星”和国防现代化做出了巨大贡献。

改革开放以来,学校以“敢为天下先”的精神,大胆推进改革:率先组成教授代表团访问美国,率先实行校内管理体制改革,率先接受海外友人巨资捐赠等,有力地推动了学校的教学科研改革。1984年,邓小平同志亲切接见了学校领导和师生代表,对学校的各项改革给予了充分肯定。在国家和上海市的大力支持下,学校以“上水平、创一流”为目标,以学科建设为龙头,先后恢复和兴建了理科、管理学科、生命学科、法学和人文学科等。1999年,上海农学院并入;2005年,与上海第二医科大学强强合并。至此,学校完成了综合性大学的学科布局。近年来,通过国家“985工程”和“211工程”的建设,学校高层次人才日渐汇聚,科研实力快速提升,实现了向研究型大学的转变。与此同时,学校通过与美国密西根大学等世界一流大学的合作办学,实施国际化战略取得重要突破。1985年开始闵行校区建设,历经20多年,已基本建设成设施完善,环境优美的现代化大学校园,并已完成了办学重心向闵行校区的转移。学校现有徐汇、闵行、法华、七宝和重庆南路(卢湾)5个校区,总占地面积4840亩。通过一系列的改革和建设,学校的各项办学指标大幅度上升,实现了跨越式发展,整体实力显著增强,为建设世界一流大学奠定了坚实的基础。

交通大学始终把人才培养作为办学的根本任务。一百多年来,学校为国家和社会培养了20余万各类优秀人才,包括一批杰出的政治家、科学家、社会活动家、实业家、工程技术专家和医学专家,如江泽民、陆定一、丁关根、汪道涵、钱学森、吴文俊、徐光宪、张光斗、黄炎培、邵力子、李叔同、蔡锷、邹韬奋、陈敏章、王振义、陈竺等。在中国科学院、中国工程院院士中,有200余位交大校友;在国家23位“两弹一星”功臣中,有6位交大校友;在18位国家最高科学技术奖获得者中,有3位来自交大。交大创造了中国近现代发展史上的诸多“第一”:中国最早的内燃机、最早的电机、最早的中文打字机等;新中国第一艘万吨轮、第一艘核潜艇、第一艘气垫船、第一艘水翼艇、自主设计的第一代战斗机、第一枚运载火箭、第一颗人造卫星、第一例心脏二尖瓣分离术、第一例成功移植同种原位肝手术、第一例成功抢救大面积烧伤病人手术等,都凝聚着交大师生和校友的心血智慧。改革开放以来,一批年轻的校友已在世界各地、各行各业崭露头角。

截至2011年12月31日,学校共有24个学院/直属系(另有继续教育学院、技术学院和国际教育学院),19个直属单位,12家附属医院,全日制本科生16802人、研究生24495人(其中博士研究生5059人);有专任教师2979名,其中教授835名;中国科学院院士15名,中国工程院院士20名,中组部“千人计划”49名,“长江学者”95名,国家杰出青年基金获得者80名,国家重点基础研究发展计划(973计划)首席科学家24名,国家重大科学研究计划首席科学家9名,国家基金委创新研究群体6个,教育部创新团队17个。

学校现有本科专业68个,涵盖经济学、法学、文学、理学、工学、农学、医学、管理学和艺术等九个学科门类;拥有国家级教学及人才培养基地7个,国家级校外实践教育基地5个,国家级实验教学示范中心5个,上海市实验教学示范中心4个;有国家级教学团队8个,上海市教学团队15个;有国家级教学名师7人,上海市教学名师35人;有国家级精品课程46门,上海市精品课程117门;有国家级双语示范课程7门;2001、2005和2009年,作为第一完成单位,共获得国家级教学成果37项、上海市教学成果157项。

\keywords{\large 上海交大 \quad 饮水思源 \quad 爱国荣校}
\end{abstract}



%% 目录、插图目录、表格目录
\tableofcontents
\listoffigures
\addcontentsline{toc}{chapter}{\listfigurename} %将插图目录加入全文目录
\listoftables
\addcontentsline{toc}{chapter}{\listtablename}  %将表格目录加入全文目录
\listofalgorithms
\addcontentsline{toc}{chapter}{\listalgorithmname} %将算法目录加入全文目录

\include{tex/symbol} % 主要符号、缩略词对照表

\mainmatter	% 使用阿拉伯数字对正文编号

%% 正文内容
\pagestyle{main}
\include{tex/intro}
\include{tex/example}
\include{tex/faq}
\include{tex/summary}

\appendix	% 使用英文字母对附录编号,重新定义附录中的公式、图图表编号样式
\renewcommand\theequation{\Alph{chapter}--\arabic{equation}}	
\renewcommand\thefigure{\Alph{chapter}--\arabic{figure}}
\renewcommand\thetable{\Alph{chapter}--\arabic{table}}
\renewcommand\thealgorithm{\Alph{chapter}--\arabic{algorithm}}

%% 附录内容,本科学位论文可以用翻译的文献替代。
%# -*- coding: utf-8-unix -*-
\chapter{搭建模板编译环境}

\section{安装TeX发行版}

\subsection{Mac OS X}

Mac用户可以从MacTeX主页\footnote{\url{https://tug.org/mactex/}}下载MacTeX 2015。
也可以通过brew包管理器\footnote{\url{http://caskroom.io}}安装MacTeX 2015。

\begin{lstlisting}[basicstyle=\small\ttfamily, numbers=none]
brew cask install mactex
\end{lstlisting}

\subsection{Linux}

建议Linux用户使用TeXLive主页\footnote{\url{https://www.tug.org/texlive/}}的脚本来安装TeXLive 2015。
以下命令将把TeXLive发行版安装到当前用户的家目录下。
若计划安装一个供系统上所有用户使用的TeXLive,请使用root账户操作。

\begin{lstlisting}[basicstyle=\small\ttfamily, numbers=none]
wget http://mirror.ctan.org/systems/texlive/tlnet/install-tl-unx.tar.gz
tar xzvpf install-tl-unx.tar.gz
cd install-tl-20150411/
./install-tl
\end{lstlisting}

\section{安装中文字体}

\subsection{Mac OS X、Deepin}

Mac和Deepin用户双击字体文件即可安装字体。

\subsection{RedHat/CentOS用户}

RedHat/CentOS用户请先将字体文件复制到字体目录下,调用fc-cache刷新缓存后即可在TeXLive中使用新字体。

\begin{lstlisting}[basicstyle=\small\ttfamily, numbers=none]
mkdir ~/.fonts
cp *.ttf ~/.fonts				# 当前用户可用新字体
cp *.ttf /usr/share/fonts/local/	# 所有用户可以使用新字体
fc-cache -f
\label{last}
\end{lstlisting}


\include{tex/app_eq}
\include{tex/app_cjk}
%# -*- coding: utf-8-unix -*-
\chapter{模板更新记录}
\label{chap:updatelog}

\textbf{2016年12月} v0.9.5发布,改用GB7714-2015参考文献风格。

\textbf{2016年11月} v0.9.4发布,增加算法和流程图。

\textbf{2015年6月19日} v0.9发布,适配ctex 2.x宏包,需要使用TeXLive 2015编译。

\textbf{2015年3月15日} v0.8发布,使用biber/biblatex组合替代 \BibTeX ,带来更强大稳定的参考文献处理能力;添加enumitem宏包增强列表环境控制能力;完善宏包文字描述。

\textbf{2015年2月15日} v0.7发布,增加盲审选项,调用外部工具插入扫描件。

\textbf{2015年2月14日} v0.6.5发布,修正一些小问题,缩减git仓库体积,仓库由sjtu-thesis-template-latex更名为SJTUThesis。

\textbf{2014年12月17日} v0.6发布,学士、硕士、博士学位论文模板合并在了一起。

\textbf{2013年5月26日} v0.5.3发布,更正subsubsection格式错误,这个错误导致如"1.1 小结"这样的标题没有被正确加粗。

\textbf{2012年12月27日} v0.5.2发布,更正拼写错误。在diss.tex加入ack.tex。

\textbf{2012年12月21日} v0.5.1发布,在 \LaTeX 命令和中文字符之间留了空格,在Makefile中增加release功能。

\textbf{2012年12月5日} v0.5发布,修改说明文件的措辞,更正Makefile文件,使用metalog宏包替换xltxtra宏包,使用mathtools宏包替换amsmath宏包,移除了所有CJKtilde(\verb+~+)符号。

\textbf{2012年5月30日} v0.4发布,包含交大学士、硕士、博士学位论文模板。模板在\href{https://github.com/weijianwen/sjtu-thesis-template-latex}{github}上管理和更新。

\textbf{2010年12月5日} v0.3a发布,移植到 \XeTeX/\LaTeX 上。

\textbf{2009年12月25日} v0.2a发布,模板由CASthesis改名为sjtumaster。在diss.tex中可以方便地改变正文字号、切换但双面打印。增加了不编号的一章“全文总结”。
添加了可伸缩符号(等号、箭头)的例子,增加了长标题换行的例子。

\textbf{2009年11月20日} v0.1c发布,增加了Linux下使用ctex宏包的注意事项、.bib条目的规范要求,
修正了ctexbook与listings共同使用时的断页错误。

\textbf{2009年11月13日} v0.1b发布,完善了模板使用说明,增加了定理环境、并列子图、三线表格的例子。

\textbf{2009年11月12日} 上海交通大学硕士学位论文 \LaTeX 模板发布,版本0.1a。


\backmatter	% 文后无编号部分 

%% 参考资料
\printbibliography[heading=bibintoc]

%% 致谢、发表论文、申请专利、参与项目、简历
%% 用于盲审的论文需隐去致谢、发表论文、申请专利、参与的项目
\makeatletter

%%
% "研究生学位论文送盲审印刷格式的统一要求"
% http://www.gs.sjtu.edu.cn/inform/3/2015/20151120_123928_738.htm

% 盲审删去删去致谢页
\ifsjtu@review\relax\else
  \include{tex/ack} 	  %% 致谢
\fi

\ifsjtu@bachelor
  % 学士学位论文要求在最后有一个英文大摘要,单独编页码
  \pagestyle{biglast}
  \include{tex/end_english_abstract}
\else
  % 盲审论文中,发表学术论文及参与科研情况等仅以第几作者注明即可,不要出现作者或他人姓名
  \ifsjtu@review\relax
    \include{tex/pubreview}
    \include{tex/projectsreview}  
  \else
    \include{tex/pub}	      %% 发表论文
    \include{tex/projects}  %% 参与的项目
  \fi
\fi

% \include{tex/patents}	  %% 申请专利
% \include{tex/resume}	  %% 个人简历

\makeatother

\end{document}
